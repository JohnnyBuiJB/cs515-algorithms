\documentclass[12pt,article]{article}
\usepackage{fullpage}
\usepackage[top=2cm, bottom=4.5cm, left=2cm, right=2cm]{geometry}
\usepackage{amsmath,amsthm,amsfonts,amssymb,amscd}
\usepackage{lastpage}
\usepackage{enumerate}
\usepackage{fancyhdr}
\usepackage{mathrsfs}
\usepackage{xcolor}
\usepackage{graphicx}
\usepackage{listings}
\usepackage{hyperref}
\usepackage{mdframed}
\usepackage{changepage}   % for the adjustwidth environment
\usepackage{forest} 
\usepackage{tikz}   % For graph

\usepackage{float}  % To inforce inserting images at the right place

\usepackage{algorithm}
\usepackage{algpseudocode}

% For recursive formulation, adapted from https://tex.stackexchange.com/questions/580333/typing-sequences-recursively-in-overleaf
\usepackage{mathtools}
\makeatletter
\newcases{centercases}{\quad}
  {\hfil$\m@th\displaystyle{##}$\hfil}
  {$\m@th\displaystyle{##}$\hfil}{\lbrace}{.}
\makeatother

\newcommand{\Tau}{\mathrm{T}}


% For matrix
\def\horzbar{\text{magic}}

\hypersetup{%
  colorlinks=true,
  linkcolor=blue,
  linkbordercolor={0 0 1}
}

\setlength{\parindent}{0.0in}
\setlength{\parskip}{0.05in}

\newcommand\projnumber{2}
\newcommand\course{CS534}
\newcommand\OSUID{934370552}
\newcommand\Email{buivy@oregonstate.edu}
\newcommand\Name{Vy Bui}
\newcommand\tab[1][1cm]{\hspace*{#1}}

\pagestyle{fancyplain}
\headheight 35pt
\lhead{Practice Assignment \projnumber}
\rhead{Jan. 30, 2023}
\lfoot{}
\cfoot{}
\rfoot{\small\thepage}
\headsep 1.5em

\newenvironment{problem}[2][Problem]
    { \begin{mdframed}[backgroundcolor=gray!20] \textbf{#1 #2} \\}
    {  \end{mdframed}}
   

% Make Rightarrow with superscript
% \makeatletter
% \newcommand{\xRightarrow}[2][]{\ext@arrow 0359\Rightarrowfill@{#1}{#2}}
% \makeatother

\begin{document}

\begin{titlepage}
    \begin{center}
        \vspace*{4cm}

        \textbf{\Large CS515 - Algorithms \& Data Structures}

        \vspace{0.5cm}
 
        \textbf{\Large Practice Assignment 1}
 
        \vspace{1cm}

        Vy Bui - 934370552

        \vspace{2cm}

        Instructor: Professor Glencora Borradaile
        \vfill
             
        \vspace{0.8cm}
      
             
        The School of Electrical Engineering and Computer Science\\
        Oregon State University\\
             
    \end{center}
\end{titlepage}

%==============================================================================%
\begin{problem}{1} 
A fixed point of an array A[1..n] is an index i such that A[i] = i. Given a sorted array of distinct integers A[1..n] as input, give a divide-and-conquer algorithm to determine if A has a fixed point that runs in time $O(log n)$.
\end{problem}

\textbf{Description}
Let $FP(i,j)$ be the function that checks if there exists a fixed point in A[i,j]. Observe that if there exists a fixed point k A[i,j], then k must be either in the left half or the right half of A[i,j]. Instead of checking the entire array, we can check its two halves and then combine the results.

\textbf{Recurrence}

\small{
    \[
    FP(i,j)=
    \begin{centercases}
        \begin{centercases}
            True               & A[i] = i \\
            False              & otherwise \\
        \end{centercases}            & i = j \\
        \begin{centercases}
            FP(i, \frac{i+j}{2}) \lor FP(\frac{i+j}{2} + 1, j)    &  \\
        \end{centercases}            & otherwise \\
    \end{centercases} 
    \]
}

\textbf{Pseudocode}

\begin{algorithm}
\caption{$FP(i,j)$}\label{alg:q2}
\begin{algorithmic}
    \If{$i == j$}
        \If{$A[i] == i$}

            \Return True
        \Else 

            \Return False
        \EndIf
    \EndIf

    \State $m \gets \frac{i+j}{2}$

    \Return $FP(i,m) \lor FP(m+1,j)$
\end{algorithmic}
\end{algorithm}

\textbf{Proof of Correctness}

\textit{Base Case}: when A has only one element, it is obvious that 


\textit{Inductive Hypothesis}: assume that we know the results of $FP(i,m)$ and $FP(m+1,j)$, with $m = \frac{i+j}{2}$.

\textit{Inductive Step}: It is trivial that if at least one of $FP(i,m)$ and $FP(m+1,j)$ is true, then $FP(i,j)$ is true because they use the same indices and values of A.

\textbf{Runing Time Analysis}

The algorithm splits the problem into two roughly equal halves and merge the results in constant time. Therefore, the running time of this algorithm is $T(n) = 2T(\frac{n}{2}) + O(1) = O(n)$

%==============================================================================%
\newpage
\begin{problem}{2} 
For a sequence of n numbers $a_1, .., a_n$, a \textit{significant inversion} is a pair $(a_i, a_j)$ such that $i < j$ and $a_i > 2a_j$ . Assuming each of the numbers $a_i$ is distinct, give an $O(nlogn)$ time algorithm to count the number of significant inversions in a sequence. (Hint: modify merge sort.)
\end{problem}

\textbf{Description}
Let $FP(i,j)$ be the function that checks if there exists a fixed point in A[i,j]. Observe that if there exists a fixed point k A[i,j], then k must be either in the left half or the right half of A[i,j]. Instead of checking the entire array, we can check its two halves and then combine the results.

\textbf{Recurrence}

\small{
    \[
    FP(i,j)=
    \begin{centercases}
        \begin{centercases}
            True               & A[i] = i \\
            False              & otherwise \\
        \end{centercases}            & i = j \\
        \begin{centercases}
            FP(i, \frac{i+j}{2}) \lor FP(\frac{i+j}{2} + 1, j)    &  \\
        \end{centercases}            & otherwise \\
    \end{centercases} 
    \]
}

\textbf{Pseudocode}

\begin{algorithm}
\caption{$FP(i,j)$}\label{alg:q3}
\begin{algorithmic}
    \If{$i == j$}
        \If{$A[i] == i$}

            \Return True
        \Else 

            \Return False
        \EndIf
    \EndIf

    \State $m \gets \frac{i+j}{2}$

    \Return $FP(i,m) \lor FP(m+1,j)$
\end{algorithmic}
\end{algorithm}

\textbf{Proof of Correctness}

\textit{Base Case}
\textit{Inductive Hypothesis}
\textit{Inductive Step}

\textbf{Runing Time Analysis}

%==============================================================================%
\newpage
\begin{problem}{3} 
You are given two sorted arrays of size m and n. Give an $O(logm + logn)$ time algorithm for computing the k-th smallest element in the union of the two arrays.
\end{problem}

\textbf{Description}
Let $FP(i,j)$ be the function that checks if there exists a fixed point in A[i,j]. Observe that if there exists a fixed point k A[i,j], then k must be either in the left half or the right half of A[i,j]. Instead of checking the entire array, we can check its two halves and then combine the results.

\textbf{Recurrence}

\small{
    \[
    FP(i,j)=
    \begin{centercases}
        \begin{centercases}
            True               & A[i] = i \\
            False              & otherwise \\
        \end{centercases}            & i = j \\
        \begin{centercases}
            FP(i, \frac{i+j}{2}) \lor FP(\frac{i+j}{2} + 1, j)    &  \\
        \end{centercases}            & otherwise \\
    \end{centercases} 
    \]
}

\textbf{Pseudocode}

\begin{algorithm}
\caption{$FP(i,j)$}\label{alg:q4}
\begin{algorithmic}
    \If{$i == j$}
        \If{$A[i] == i$}

            \Return True
        \Else 

            \Return False
        \EndIf
    \EndIf

    \State $m \gets \frac{i+j}{2}$

    \Return $FP(i,m) \lor FP(m+1,j)$
\end{algorithmic}
\end{algorithm}

\textbf{Proof of Correctness}

\textit{Base Case}
\textit{Inductive Hypothesis}
\textit{Inductive Step}


\textbf{Runing Time Analysis}

%==============================================================================%
\newpage
\begin{problem}{4} 
You are given an $n \times n$ matrix A[1..n, 1..n] where all elements are distinct. We say that an element A[x] is a \textit{local minimum} if it is less than its (at most) four neighbors, i.e. its up, down, left and right neighbors. Give an $O(n)$ time algorithm to find a local minimum of A.
\end{problem}

\textbf{Description}
Let $FP(i,j)$ be the function that checks if there exists a fixed point in A[i,j]. Observe that if there exists a fixed point k A[i,j], then k must be either in the left half or the right half of A[i,j]. Instead of checking the entire array, we can check its two halves and then combine the results.

\textbf{Recurrence}

\small{
    \[
    FP(i,j)=
    \begin{centercases}
        \begin{centercases}
            True               & A[i] = i \\
            False              & otherwise \\
        \end{centercases}            & i = j \\
        \begin{centercases}
            FP(i, \frac{i+j}{2}) \lor FP(\frac{i+j}{2} + 1, j)    &  \\
        \end{centercases}            & otherwise \\
    \end{centercases} 
    \]
}

\textbf{Pseudocode}

\begin{algorithm}
\caption{$FP(i,j)$}\label{alg:q1-fp}
\begin{algorithmic}
    \If{$i == j$}
        \If{$A[i] == i$}

            \Return True
        \Else 

            \Return False
        \EndIf
    \EndIf

    \State $m \gets \frac{i+j}{2}$

    \Return $FP(i,m) \lor FP(m+1,j)$
\end{algorithmic}
\end{algorithm}

\textbf{Proof of Correctness}

\textit{Base Case}
\textit{Inductive Hypothesis}
\textit{Inductive Step}

\textbf{Runing Time Analysis}

\bibliographystyle{alpha}
\bibliography{mybib}
\end{document}
