\documentclass[12pt,article]{article}
\usepackage{fullpage}
\usepackage[top=2cm, bottom=4.5cm, left=2cm, right=2cm]{geometry}
\usepackage{amsmath,amsthm,amsfonts,amssymb,amscd}
\usepackage{lastpage}
\usepackage{enumerate}
\usepackage{fancyhdr}
\usepackage{mathrsfs}
\usepackage{xcolor}
\usepackage{graphicx}
\usepackage{listings}
\usepackage{hyperref}
\usepackage{mdframed}
\usepackage{changepage}   % for the adjustwidth environment
\usepackage{forest} 
\usepackage{tikz}   % For graph

\usepackage{float}  % To inforce inserting images at the right place

\usepackage{algorithm}
\usepackage{algpseudocode}

% For recursive formulation, adapted from https://tex.stackexchange.com/questions/580333/typing-sequences-recursively-in-overleaf
\usepackage{mathtools}
\makeatletter
\newcases{centercases}{\quad}
  {\hfil$\m@th\displaystyle{##}$\hfil}
  {$\m@th\displaystyle{##}$\hfil}{\lbrace}{.}
\makeatother

\newcommand{\Tau}{\mathrm{T}}


% For matrix
\def\horzbar{\text{magic}}

\hypersetup{%
  colorlinks=true,
  linkcolor=blue,
  linkbordercolor={0 0 1}
}

\setlength{\parindent}{0.0in}
\setlength{\parskip}{0.05in}

\newcommand\projnumber{3}
\newcommand\course{CS534}
\newcommand\OSUID{934370552}
\newcommand\Email{buivy@oregonstate.edu}
\newcommand\Name{Vy Bui}
\newcommand\tab[1][1cm]{\hspace*{#1}}

\pagestyle{fancyplain}
\headheight 35pt
\lhead{Practice Assignment \projnumber}
\rhead{Feb. 20, 2023}
\lfoot{}
\cfoot{}
\rfoot{\small\thepage}
\headsep 1.5em

\newenvironment{problem}[2][Problem]
    { \begin{mdframed}[backgroundcolor=gray!20] \textbf{#1 #2} \\}
    {  \end{mdframed}}
   

% Make Rightarrow with superscript
% \makeatletter
% \newcommand{\xRightarrow}[2][]{\ext@arrow 0359\Rightarrowfill@{#1}{#2}}
% \makeatother

\begin{document}

\begin{titlepage}
    \begin{center}
        \vspace*{4cm}

        \textbf{\Large CS515 - Algorithms \& Data Structures}

        \vspace{0.5cm}
 
        \textbf{\Large Practice Assignment \projnumber}
 
        \vspace{1cm}

        Vy Bui - 934370552

        \vspace{2cm}

        Instructor: Professor Glencora Borradaile
        \vfill
             
        \vspace{0.8cm}
      
             
        The School of Electrical Engineering and Computer Science\\
        Oregon State University\\
             
    \end{center}
\end{titlepage}

%==============================================================================%
\begin{problem}{1} 
Job Scheduling
\end{problem}

\textbf{(a)}
Let $T = \{t_1, t_2, ..., t_n\}$ be the time the jobs $j_1, j_2, ..., j_n$ take.

\begin{algorithm}
\caption{$A(T)$}\label{alg:q1}
\begin{algorithmic}
    \State $sortedT \gets sort(T)$
    \State $lastJobCompleteTime \gets 0$
    \State $totalTime \gets 0$
    \For{t in T}
        \State $totalTime \gets totalTime + lastJobCompleteTime + t$
        \State $lastJobCompleteTime \gets lastJobCompleteTime + t$
    \EndFor

    \Return $sortT, totalTime$
\end{algorithmic}
\end{algorithm}

\textbf{(b)}
We have 
$$\sum_{i=1}^{n} C_i = t_1 + (t_1 + t_2) + (t_1 + t_2 + t_3) + ... + t_n 
= nt_1 + (n-1)t_2 + ... + (n+1-i)t_i + (n-i)t_{i+1} + ... + t_n$$

Theorem 1: The total cost is minimum when $t_i \leq t_{i+1}$ for all i \\
Proof: assume that there exists some optimal job ordering that has $t_i > t_{i+1}$. Observe that there are (n + 1 - i) of $t_i$ terms and (n - i) of $t_{i+1}$ terms in the above summation. If we swap $t_i$ and $t_{i+1}$, the summation will have one more $t_{i+1}$ term and one less $t_i$ term. Because $t_i > t_{i+1}$, the total cost after the swap will reduce, thus producing a not worse solution. From some optimal solution O, we can swap these inversions ($t_{i} > t_{i+1}$) until there is no inversions left in the ordering, which is exactly the solution of our greedy algorithm. And each swap guarantees to produce at least equally good result.

\textbf{(c)} \\
The algorithm takes $O(nlogn)$ to sort the list of jobs by time needed to complete the job. It then takes $O(n)$ time to iterate through the sorted list and accumulate the total time. The ordering is the order of the sorted list. In total, it takes $O(nlogn)$ time.

%==============================================================================%
\newpage
\begin{problem}{2} 
A wrong greedy algorithm for the Knapsack problem 
\end{problem}

\textbf{(a)}
First, we find the list of bricks $C$ that can be fit in the bags $c1, c2, ..., c_{i}$, then we iteratively take the heaviest brick in $C$ until the bag is full.

Assume that B is the list of bricks sorted by weights.
\begin{algorithm}
\caption{$A(B,n)$}\label{alg:q2}
\begin{algorithmic}
    \State $w \gets 0$
    \While{$n > 0$}
        \State Find the heaviest brick $b < n$
        \State $w \gets w + b$
        \State $n \gets n - b$
    \EndWhile

    \Return $w$
\end{algorithmic}
\end{algorithm}

\textbf{(b)} \\
Observe that the brick weights are equivalent to binary representation $2^0,2^1,2^2,...$. Our algorithm is analogous to finding the binary representation of $n$. It starts with the most significant bit of n, then reduce n to, and repeat that process until n is reduced to 0. Because every number has a binary representation, the algorithm guarantees to fill up any given bag (completely full), hence produce the optimal solution.

% Let $i = 1,2,..,n$ is the order in which the gold bricks are put in the bag.

% Theorem 1: the maximum number of gold bricks g that the bag can carry is the number of the lightest gold bricks we can put in the bag.

% Proof: Let $G$ be the list of the g-lighest bricks. Assume that there exists some optimal solution O that contains some bricks $b_i \notin G$. To be able to take $b_i$, O must have left out some bricks $b_j \in G$, otherwise, there would not be enough room for $b_i + G$ (by the assumption that G-lightest bricks fill up the bag). If we swap $b_i$ with $b_j$, we get a new bag with the same number of gold bricks but with less weight. By swapping out all of the large bricks, $b_i \notin G$, with some bricks, $b_j \in G$, we obtain the greedy algorithm's result. And each swap results in an at least equally good solution. Therefore, the greedy solution is optimal.

\textbf{(c)} \\
With gold bricks having weights $\{2,2,3,5\}$ and $n = 9$, the algorithm will not produce the optimal solution. In particular, the optimal solution is $\{2,2,5\}$, while the algorithm will yield $\{5,3\}$.

%==============================================================================%
\newpage
\begin{problem}{3} 
A randomized algorithm for generating biased random bits
\end{problem}

\textbf{(a)} \\
Let $P(1) = p$ and $P(0) = 1 - p$ denote the probability that ONEINTHREE returns 1 and 0, respectively. Let F denote the probability distribution of FAIRCOIN, with $F(1) = F(0) = 0.5$.

We have 
$$P(0) = 0.5 + (1 - 0.5)P(1) = 0.5 + 0.5P(1) \tab[1cm]$$  

We also have
$$P(1) = 1 - P(0) = 1 - 0.5 - 0.5P(1) = 0.5 - 0.5P(1) \tab[1cm]$$ 

Hence, $P(1) = \frac{1}{3}$

\textbf{(b)} \\
The function only terminates when FAIRCOIN returns 0. The expected number of times we need to call FAIRCOIN it returns 0 is $\frac{1}{0.5} = 2$.

\textbf{(c)}

\begin{algorithm}
\caption{$FAIRCOIN$}\label{alg:q3}
\begin{algorithmic}
    \If{$ONEINTHREE == 1$}
        \Return 1
    \Else
        \Return $(1 - FAIRCOIN)(1 - FAIRCOIN)$
    \EndIf
\end{algorithmic}
\end{algorithm}

We have 
$$F(1) = P(1) + P(0)F(0)F(0) \tab[1cm] (1)$$  

We also have
$$F(0) = 1 - F(1) \tab[1cm] (2)$$ 

Plugging (1) into (2) results in
$$F(0) = \frac{2}{3} - \frac{2}{3}F^2(0) \tab[1cm] (3)$$ 

Solving (3) we get $F(0) = 0.5$.


%==============================================================================%
\newpage
\begin{problem}{4} 
Tax Screening System
\end{problem}

Chernoff inequality: \\
$$Pr[X > (1 + \delta)\mu] < (\frac{e^{\delta}}{(1+\delta)^{1+\delta}})^{\mu}$$

$$Pr[X > (1 - \delta)\mu] < e^{-\frac{1}{2} \mu \delta^2}$$
\bibliographystyle{alpha}
\bibliography{mybib}
\end{document}
