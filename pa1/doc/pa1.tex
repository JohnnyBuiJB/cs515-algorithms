\documentclass[12pt,article]{article}
\usepackage{fullpage}
\usepackage[top=2cm, bottom=4.5cm, left=2cm, right=2cm]{geometry}
\usepackage{amsmath,amsthm,amsfonts,amssymb,amscd}
\usepackage{lastpage}
\usepackage{enumerate}
\usepackage{fancyhdr}
\usepackage{mathrsfs}
\usepackage{xcolor}
\usepackage{graphicx}
\usepackage{listings}
\usepackage{hyperref}
\usepackage{mdframed}
\usepackage{changepage}   % for the adjustwidth environment
\usepackage{forest} 
\usepackage{tikz}   % For graph

\usepackage{float}  % To inforce inserting images at the right place

\usepackage{algorithm}
\usepackage{algpseudocode}

% For recursive formulation, adapted from https://tex.stackexchange.com/questions/580333/typing-sequences-recursively-in-overleaf
\usepackage{mathtools}
\makeatletter
\newcases{centercases}{\quad}
  {\hfil$\m@th\displaystyle{##}$\hfil}
  {$\m@th\displaystyle{##}$\hfil}{\lbrace}{.}
\makeatother

\newcommand{\Tau}{\mathrm{T}}


% For matrix
\def\horzbar{\text{magic}}

\hypersetup{%
  colorlinks=true,
  linkcolor=blue,
  linkbordercolor={0 0 1}
}

\setlength{\parindent}{0.0in}
\setlength{\parskip}{0.05in}

\newcommand\projnumber{1}
\newcommand\course{CS534}
\newcommand\OSUID{934370552}
\newcommand\Email{buivy@oregonstate.edu}
\newcommand\Name{Vy Bui}
\newcommand\tab[1][1cm]{\hspace*{#1}}

\pagestyle{fancyplain}
\headheight 35pt
\lhead{Practice Assignment \projnumber}
\rhead{Jan. 19, 2023}
\lfoot{}
\cfoot{}
\rfoot{\small\thepage}
\headsep 1.5em

\newenvironment{problem}[2][Problem]
    { \begin{mdframed}[backgroundcolor=gray!20] \textbf{#1 #2} \\}
    {  \end{mdframed}}
   

% Make Rightarrow with superscript
% \makeatletter
% \newcommand{\xRightarrow}[2][]{\ext@arrow 0359\Rightarrowfill@{#1}{#2}}
% \makeatother

\begin{document}

\begin{titlepage}
    \begin{center}
        \vspace*{4cm}

        \textbf{\Large CS515 - Algorithms \& Data Structures}

        \vspace{0.5cm}
 
        \textbf{\Large Practice Assignment 1}
 
        \vspace{1cm}

        Vy Bui - 934370552

        \vspace{2cm}

        Instructor: Professor Glencora Borradaile
        \vfill
             
        \vspace{0.8cm}
      
             
        The School of Electrical Engineering and Computer Science\\
        Oregon State University\\
             
    \end{center}
\end{titlepage}

%==============================================================================%
\begin{problem}{1} 
Suppose you are given an array A[1..n] of integers, which may be positive, negative, or zero. Describe a linear-time (i.e. O(n)-time) algorithm that finds the largest sum of elements in a contiguous, nonempty subarray A[i..j]. For example, given the array [-6, 12,-7, 0, 14,-7, 5] as input, your algorithm should return the integer 19 (the sum of [12,-7, 0, 14]).

For the sake of analysis, assume that comparing, adding, or multiplying any pair of numbers takes O(1) time.

\end{problem}

\textbf{Recursive Formulation}

First, let $MS(k)$ denote the function that find the nonempty contiguous subarray with the largest sum in $A[k:n]$. Second, let $MP(k)$ finds the contiguous subarray that starts from $k$ and has the largest sum. Note that $MPs$ can be empty. For example, let $A = [-3,-1,2,1]$, $MS(2) = 3, MP(2) = 2$, with the two last elements as MS and the last three elements as MP. However, $MS(1) = 3, MP(1) = 0$ with the two last elements as MS and the empty MP.

$MP$ and $MS$ can be defined recursively as follows:

% \small{
%     \[
%         LSA(k)=
%         \begin{centercases}
%             (A[k], max(0, A[k]))               & k = n \\
%             (0,0)               & k > n \lor k \leq 0 \\
%             (max(A[k], A[k] + LSA(k+1)[1], LSA(k+1)[0]),                &otherwise \\
%             max(0, LSA(k+1)[1]))        & 
%         \end{centercases}
%         \]
% }

\small{
    \[
        MP(k)=
        \begin{centercases}
            0                          & k > n \lor k \leq 0 \\
            max(0, A[k])               & k = n \\
            max(0, A[k] + MP(k+1))        & otherwise
        \end{centercases}
        \]
}

\small{
    \[
        MS(k)=
        \begin{centercases}
            0                  & k > n \lor k \leq 0 \\
            A[k]               & k = n \\
            max(A[k], A[k] + MP(k+1), MS(k+1))     & otherwise \\
        \end{centercases}
        \]
}

\normalsize{}

\textbf{Proof/Explanation}

Proof 1a. For MP's recursive fomular, the first line prevents the algorithm to get out of the boundary of the subject array. In the base case on the second line, there is only one element in $A[k:k]$, so the result must be either empty or $A[k]$. For the inductive step, assume that $MP(k+1)$ returns the max prefix (contiguous subarray that starts from $k+1$ and has the largest sum) of $A[k]$, then $MP[k]$ must be $max(0, A[k] + MP(k+1))$. Because if there exists a larger prefix $A[k:j]$ so that $sum(A[k:j]) = A[k] + sum(A[k+1:j]) > A[k] + MP(k+1)$, then $sum(A[k+1:j] > MP(k+1))$, which contradicts the assumption that $MP(k+1)$ is the max prefix of $A[k+1:n]$. Proof completes.

Proof 1b. For MS's recursive formula, the first line prevents the algorithm to get out of the boundary of the subject array. The second line is the base case in which we consider one element at the right end of the array. $A[k]$ is the only non-empty subarray in $A[k:k]$ so it must also be MS of $A[k:k]$. For the inductive step, assume that $MS(k+1)$ returns the max subarray of $A[k+1:n]$ and $MS(k+1)$ returns that max prefix, $MS(k)$ is guaranteed to be the maximum of the three case mentioned in the third line of the $MS(k)$ formula. Because if there exists a better solution, that is a contiguous subarray whose sum is greater than all of these cases, it will either contain $A[k]$ or not. In the first case when the subarray $S1$ does not contain $A[k]$, it must be within $A[k+1:n]$. $sum(S1) > MS(k+1)$ contradicts the assumption that $MS(k+1)$ is the max subarray of $A[k+1:n]$, hence there not exists such an subarray. In the second case where the subarray $S2$ contains $A[k]$, it can be decomposed to $sum(S2) = A[k] + sum(A[k+1:j])$. Again, $sum(S2) > A[k] + MP(k+1) \iff sum(A[k+i:j]) > MP(k+1)$ contradicts the assumption that $MP(k+1)$ is the max prefix, thus there not exists such an subarray. Proof completes!

\newpage
\textbf{Pseudocode}
Because $MS(k)$ only depends on $MS(k+1)$ and $MP(k)$ only depends on $MP(k+1)$, we can implement them iteratively in the same loop.

\begin{algorithm}
\caption{$(A[1:n])$}\label{alg:LSA}
\begin{algorithmic}
    \State $ms \gets A[k]$\
    \State $mp \gets max(0, A[k])$\
    \State $k \gets n - 1$\;
    
    \While{$k \geq 1$}
        \State $mp \gets mp + A[k]$

        \State $max\_sum \gets max(A[k], ms, mp)$\

        \If{$mp == max\_sum$}
            \State $ms \gets mp$\
        \ElsIf{$A[k] == max\_sum$}
            \State $ms \gets A[k]$\
            \State $mp \gets A[k]$\
        \Else
            \State $mp \gets max(0, mp)$\
        \EndIf

        \State $k \gets k - 1$
    \EndWhile

    \Return $ms$
\end{algorithmic}
\end{algorithm}

\textbf{Runing Time and Space Analysis}
There are n iterations, each of which has constant number of operations, hence the algorithm has $O(n)$ time complexity. Furthermore, the algorithm uses $O(1)$ space.

%==============================================================================%
\newpage
\begin{problem}{2} 
String A is a supersequence of string B if string B can be obtained from string A by removing letters. For example, the strings BARNYARDSNACK, YUMMYBANANAS, and BWANWANWA are supersequences of the string BANANA. Give a dynamic program for finding the length of the shortest string that is a supersequence of two input strings A and B.
\end{problem}

We can create the shortest supersequence of A and B by modifying one of them to make it become a supersequence of the other with the least modifications. To keep the resulted sequence a supersequence of the starting string, only insertions are allowed. The proof will be given in the subsequent section. 

The problem now becomes finding the least number of insertions to turn one string to another. This resembles the Edit Distance problem mentioned in \cite{JeffE19} but simpler because it only allows insertions. Intuitively, starting with a longer string leads to less insertions, but both ways will result an optimal solution.

\textbf{Recursive Formulation}

Let $MI(i,j)$ denote the program that can return the minimum of insertions to transform $A[i:n]$ to a supersequence of $B[j:m]$. 

\normalsize{
    \[
        MI(i,j)=
        \begin{centercases}
            j   & i = 0 \\
            0   & j = 0 \\
            min(MI(i,j+1) + 1, MI(i+1,j+1)) & otherwise
        \end{centercases}
        \]
}

\textbf{Proof/Explanation}

Assume that we know the minimum number of insertions k needed to transform A to a supersequence S of B. S will be also a supersequence of A because we can always remove the inserted letters to transform it back to A. To prove that S is the shortest supersequence of A and B, proof by contradiction is used. Assume that there exists a shorter supersequence R of A and B. We can always transform A to R by only inserting some m letters into A. Because R is shorter than S, m is shorter than k, which contradicts with the assumption that k is the minimum number of insertions needed to transform A to a supersequence of B. Therefore, R does not exist and S is the shortest supersequence of A and B.

\textbf{Pseudocode}

\textbf{Runing Time Analysis}

%==============================================================================%
\newpage
\begin{problem}{3} 
Find the length (number of edges) of the longest path in a binary tree.
\end{problem}

\textbf{Observation 1:} A binary tree T includes two subtrees L and R, and the root node. The longest path of T, LP(T) can be a path within L, a path within R (2), or a combination of the longest path from the root node to some node in L and the longest path from the root node to some node in R (3). The longest path from some node in L to the root node is the depth of L + 1, and the same applies to R.

\textbf{Observation 2:} The depth of tree T, D(T), is the largest of the depth of the left subtree and the depth of the right subtree.

\textbf{Recursive Formulation}
\normalsize{
\[
    D(T)=
    \begin{centercases}
        -1   & T = \emptyset \\
        0    & T.L = \emptyset \land T.R = \emptyset \\
        1 + max(D(L), D(R)) & otherwise
    \end{centercases}
\]
}

\normalsize{
\[
    LP(T)=
    \begin{centercases}
        -1   & T = \emptyset \\
        0    & T.L = \emptyset \land T.R = \emptyset \\
        max(LP(L), LP(R), D(L) + D(R) + 2) & otherwise
    \end{centercases}
\]
}

\textbf{Proof/Explanation}

We will prove that there not exist a path in T that is longer than LP(L), LP(R), and D(L) + D(R) + 2, using proof by contradiction. Assume that there exists a longer path $(x,y)$ with $x,y \in L \cup R \cup r$ where $r$ is the root node. If $x,y \in L$, then it contradicts with the assumption that LP(L) is the longest path in L, which shows that such path does not exist. The same argument can be applied to path within the right tree. Lastly, if $x \in L$ and $y \in R$ then it has to go through r. We can break the path into two pieces, $(x,r)$ and $(r,y)$ so that $|(x,y)| = |(x,r)| + |(r,y)|$, which can not be greater than D(L) + D(R). Thus, such a path does not exist.

\newpage
\textbf{Pseudocode}
For this problem, a recursive implementation seems to be more efficient, readable, and easier to implement compared to an iterative implementation.

\begin{algorithm}
\caption{$D(T)$}\label{alg:D}
\begin{algorithmic}
    \If{$T == \emptyset$}
        \Return -1
    \EndIf

    \If{$T.L == \emptyset \land T.R == \emptyset$}
        \Return 0
    \EndIf

    \Return $1 + max(D(T.L), D(T.R))$
\end{algorithmic}
\end{algorithm}

\begin{algorithm}
\caption{$LP(T)$}\label{alg:LP}
\begin{algorithmic}
    \If{$T == \emptyset$}
        \Return -1
    \EndIf

    \If{$T.L == \emptyset \land T.R == \emptyset$}
        \Return 0
    \EndIf

    \Return $max(LP(T.L), LP(T.R), D(T.L) + D(T.R) + 2)$
\end{algorithmic}
\end{algorithm}

\textbf{Runing Time Analysis}

The iterative implementation needs to compute LP from the bottom up, that is from the leaves up to the root. The problem is we do not know the leaves in advance so it needs to traverse the trees one more time to find the leaves. For memory, it needs O(n) of memory to keep the results in the worst case that the tree is a full binary tree, which has (n + 1) / 2 leaves. On the other hand, a recursive algorithm only needs to traverse through the tree once because a node depends on exactly its two children nodes. Therefore, it takes O(n) time and O(n) space of stack to store the recursive calls.

%==============================================================================%
\newpage
\begin{problem}{4} 
Suppose you are given an n x n bitmap, represented by a 2-dimensional array M[1..n, 1..n] of 0s and 1s. A solid block in M is a subarray of the form M[i..i', j..j'] containing only 1s. Describe an algorithm to find the area of the maximum solid block in M in $O(n^3)$ time. If you can do that, try to design a faster algorithm that runs in $O(n^2)$ time.
\end{problem}

\textbf{Recursive Formulation}

\textbf{Proof/Explanation}

\textbf{Pseudocode}

\textbf{Runing Time Analysis}

\bibliographystyle{alpha}
\bibliography{mybib}
\end{document}
