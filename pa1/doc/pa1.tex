\documentclass[12pt,article]{article}
\usepackage{fullpage}
\usepackage[top=2cm, bottom=4.5cm, left=2cm, right=2cm]{geometry}
\usepackage{amsmath,amsthm,amsfonts,amssymb,amscd}
\usepackage{lastpage}
\usepackage{enumerate}
\usepackage{fancyhdr}
\usepackage{mathrsfs}
\usepackage{xcolor}
\usepackage{graphicx}
\usepackage{listings}
\usepackage{hyperref}
\usepackage{mdframed}
\usepackage{changepage}   % for the adjustwidth environment
\usepackage{forest} 
\usepackage{tikz}   % For graph

\usepackage{float}  % To inforce inserting images at the right place

\usepackage{algorithm}
\usepackage{algpseudocode}

% For recursive formulation, adapted from https://tex.stackexchange.com/questions/580333/typing-sequences-recursively-in-overleaf
\usepackage{mathtools}
\makeatletter
\newcases{centercases}{\quad}
  {\hfil$\m@th\displaystyle{##}$\hfil}
  {$\m@th\displaystyle{##}$\hfil}{\lbrace}{.}
\makeatother

\newcommand{\Tau}{\mathrm{T}}


% For matrix
\def\horzbar{\text{magic}}

\hypersetup{%
  colorlinks=true,
  linkcolor=blue,
  linkbordercolor={0 0 1}
}

\setlength{\parindent}{0.0in}
\setlength{\parskip}{0.05in}

\newcommand\projnumber{1}
\newcommand\course{CS534}
\newcommand\OSUID{934370552}
\newcommand\Email{buivy@oregonstate.edu}
\newcommand\Name{Vy Bui}
\newcommand\tab[1][1cm]{\hspace*{#1}}

\pagestyle{fancyplain}
\headheight 35pt
\lhead{IA1 Competition \projnumber}
\rhead{Oct. 9, 2022}
\lfoot{}
\cfoot{}
\rfoot{\small\thepage}
\headsep 1.5em

\newenvironment{problem}[2][Problem]
    { \begin{mdframed}[backgroundcolor=gray!20] \textbf{#1 #2} \\}
    {  \end{mdframed}}
   

% Make Rightarrow with superscript
% \makeatletter
% \newcommand{\xRightarrow}[2][]{\ext@arrow 0359\Rightarrowfill@{#1}{#2}}
% \makeatother

\begin{document}

\begin{titlepage}
    \begin{center}
        \vspace*{4cm}

        \textbf{\Large CS515 - Algorithms \& Data Structures}

        \vspace{0.5cm}
 
        \textbf{\Large Practice Assignment 1}
 
        \vspace{1cm}

        Vy Bui - 934370552

        \vspace{2cm}

        Instructor: Professor Glencora Borradaile
        \vfill
             
        \vspace{0.8cm}
      
             
        The School of Electrical Engineering and Computer Science\\
        Oregon State University\\
             
    \end{center}
\end{titlepage}

%==============================================================================%
\large{\textbf{Question 1}}\newline

\normalsize{}
\textbf{Recursive Formulation}
Let $LSA(k)$ denote the function that can find the subarray of $A[k:n]$ with the largest sum of its elements. $LSA$ returns a pair of the sum of the largest subarray and its prefix sum in the aforementioned order. For example, let $A = [-1,2,1]$, $LSA(1) = (3,-1)$, with the two last elements as the largest subarray and the first element as the prefix. 


$LSA$ can be defined recursively as follows:

\tiny{
    \[
        LSA(k)=
        \begin{centercases}
            (0,0)               & k > n \land k < 1 \\
            (A[k],0)         & A[k] > LSA(k+1)[0] \land LSA(k+1)[0] < -LSA(k+1)[1] \\
            (A[k] + LSA(k+1)[0] + LSA(k+1)[1],0) & A[k] \geq -LSA(k+1)[1] \land LSA(k+1)[0] \geq -LSA(k+1)[1] \\
            (LSA(k+1)[0], A[k] + LSA(k+1)[1]) & A[k] < -LSA(k+1)[1] \land A[k] < LSA(k+1)[0] \\
        \end{centercases}
        \]
}

\normalsize{}

\textbf{Proof/Explanation}

For simplicity, in this section, let $LSA(k)$ denote the subarray of interest only, but not its prefix. This problem's optimal substructure property can be described as follows. The optimal solution to the problem with $A[k:n]$ can be found based on the optimal solution to the problem with $A[k+1:n]$. In particular, assume that we know the optimal solution to the problem of $A[k+1:n]$, adding $A[k]$ to the problem results in two new candidate largest subarrays. First, $A[k]$ can be combined with $LSA(k+1)$ to create a new solution. Note that this comes with the cost of the sum of the prefix of $LSA(k+1)$ because the new solution has to be contiguous. The second candidate is $A[k]$ itself. $LSA(k)$ has the largest sum among these two new candidates and $LSA(k+1)$. Note that the combination solution is prioritized when a tie happens in order to open up opportunities for further combination.

We can use proof by contradiction to prove this algorithm. Assume that there exists a better solution $LSA'(k)$ that is not one of the three candidates, $A[k]$, the combination, and $LSA(k+1)$. First, because $LSA'(k)$ is not $A[k]$, it must be a subarray of $A[k+1:n]$. This makes $LSA'(1)$ the subarray of the largest sum of $A[k+1:n]$ because $sum(LSA'(k)) > sum(LSA(k+1))$ based on our assumption. However, this contradicts another assumption that $LSA(2)$ is the optimal solution to the problem with $A[2:n]$. The proof completes!

\newpage
\textbf{Pseudocode}

Observe that $LSA(k)$ only depends on $LSA(k+1)$, we can implement this algorithm iteratively from $LSA(n)$ to $LSA(0)$. The solution is $LSA(0)$.


\begin{algorithm}
\caption{$LSA(A[1:n])$}\label{alg:LSA}
\begin{algorithmic}
    \State $last\_prefix\_sum \gets 0$\
    \State $last\_largest\_sum \gets 0$\
    \State $k \gets n$\;
    
    \While{$k \geq 1$}
        \State $combined\_sum \gets A[k] + last\_largest\_sum + last\_prefix\_sum$

        \State $max\_largest\_sum \gets max(A[k], last\_largest\_sum, combined\_sum)$\

        \If{$combined\_sum == max\_largest\_sum$}
            \State $last\_prefix\_sum \gets 0$\
            \State $last\_largest\_sum \gets combined\_sum$\
        \ElsIf{$A[k] == max\_largest\_sum$}
            \State $last\_prefix\_sum \gets 0$\
            \State $last\_largest\_sum \gets A[k]$\
        \ElsIf{$last\_largest\_sum == max\_largest\_sum$}
            \State $last\_prefix\_sum \gets last\_prefix\_sum + A[k]$\
        \EndIf

        \State $k \gets k - 1$
    \EndWhile

    \Return $last\_largest\_sum$
\end{algorithmic}
\end{algorithm}

\textbf{Runing Time and Space Analysis}
There are n iteration, each of which has constant number of operations, hence the algorithm has $O(n)$ time complexity. Furthermore, the algorithm uses $O(1)$ space.
%==============================================================================%
\newpage
\large{\textbf{Question 2}}\newline
\normalsize{}

\textbf{Recursive Formulation}

\textbf{Proof/Explanation}

\textbf{Pseudocode}

\textbf{Runing Time Analysis}

%==============================================================================%
\newpage
\large{\textbf{Question 3}}\newline
\normalsize{}

\textbf{Recursive Formulation}

\textbf{Proof/Explanation}

\textbf{Pseudocode}

\textbf{Runing Time Analysis}

%==============================================================================%
\newpage
\large{\textbf{Question 4}}\newline
\normalsize{}

\textbf{Recursive Formulation}

\textbf{Proof/Explanation}

\textbf{Pseudocode}

\textbf{Runing Time Analysis}


\end{document}
